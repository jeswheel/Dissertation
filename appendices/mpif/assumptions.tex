\section{Assumptions for Theorem~1}\label{sec:assumptions}

For all constants $(\epsilon, r) \in (0, \infty) \times \R^d$, we define $B^d_\epsilon(r) = \{r' \in \R^d: \vert r - r'\vert_2 < \epsilon\}$. 
Let $\BorelT$ be the Borel $\sigma$-algebra on the set of real numbers $\R^{\nshared+\Unit\nspecific}$.
We assume that $\Theta \in \BorelT$ is a compact set that satisfies \ref{assumption:regular}.
Informally, this ensures that the corners of the set are not too sharp, and directly follows definition of a regular compact set from \citet{chen24}.
\begin{enumerate}[label=(A\arabic*), ref=(A\arabic*)]
  % \item  $\Theta \in \BorelT$ is a compact set. \label{assumption:compact}
  \item \label{assumption:regular} There exists a continuous function $\regFun\colon [0, \infty) \to [0, \infty)$ such that $\lim_{x\downarrow 0} \regFun(x) = 0$, and for all $(\epsilon, x) \in (0,\infty) \times \Theta$, there exists $x' \in \Theta$ such that
$$
B^{\nshared+\Unit\nspecific}_{\regFun(\epsilon)}(x') \subseteq B^{\nshared+\Unit\nspecific}_{\epsilon}(x) \cap \Theta.
$$
\end{enumerate}
Let $\theta \in \R^{\nshared+\Unit\nspecific}$
We make the following assumptions on the probability densities that are used to define a PanelPOMP model described in Section~\ref{sec:ppomp}.
\begin{enumerate}[label=(B\arabic*), ref=(B\arabic*)]
\item $L(\theta; \bm{y}^*) > 0$ for all $\theta \in \Theta$ and $\sup_{(\theta, x_{u, n})\in (\Theta, \mathcal{X})}f_{Y_{u, n}|X_{u, n}}(y^*_{u, n}|x_{u,n}; \, \theta) < \infty$ for all $u \in \seq{1}{U}$ and $n \in \seq{1}{N_u}$. \label{assumption:mle1}
    \item The transition and measurement densities are sufficiently smooth functions of $\theta$, in the sense that for any $\theta, \theta' \in \Theta$, there exists a a continuous and strictly increasing function $g: [0, \infty) \rightarrow [0, \infty)$ and sequence of measurable functions $\varphi_{u, 0:N_u}: \mathcal{X}^2 \rightarrow \mathbb{R}$, that satisfy:
    \begin{align*}
    \begin{split}
    |&\log\big(f_{Y_{u, n}|X_{u, n}}(y^*_{u, n}|x_{u, n};\, \theta)f_{X_{u, n}|X_{u, n-1}}(x_{u, n}|x_{u, n-1};\, \theta)\big) \\
    & - \log\big(f_{Y_{u, n}|X_{u, n}}(y^*_{u, n}|x_{u, n};\, \theta')f_{X_{u, n}|X_{u, n-1}}(x_{u, n}|x_{u, n-1};\, \theta')\big)| \leq g\big(\|\theta - \theta'\|\big)\varphi_{u, n}(x_{u, n-1}, x_{u, n}),
    \end{split}
    \end{align*}
    With the constraint that for all $u\in \seq{1}{U}$, there exists a $\delta_u \in (0, \infty)$ such that
    \begin{align*}
    &  \int \exp\Big\{\delta_u \sum_{n = 0}^{N_u}\varphi_{u, n}(x_{u, n-1}, x_{u, n})\Big\} f_{X_{u, 0}}(x_{u, 0};\, \theta) \times
    \\
    & \hspace{20mm} \prod_{n = 1}^{N_u}f_{Y_{u, n}|X_{u, n}}(y^*_{u, n}|x_{u, n};\, \theta)f_{X_{u, n}|X_{u, n-1}}(x_{u, n}|x_{u, n-1};\, \theta)\, dx_{u, 0:N_u} < \infty,
    \end{align*}
    using the convention that if $n=0$, then
    \begin{eqnarray*}
      \varphi_{u, n}(x_{u, n-1}, x_{u, n}) &=& \varphi_{u, 0}(x_{u, 0}, x_{u, 0})
      \\
      f_{Y_{u, n}|X_{u, n}}(\cdot | x_{u, n}; \, \theta) &=& 1
      \\
      f_{X_{u, n} | X_{u, n-1}}(x_{u, n}|x_{u, n-1};\, \theta) &=& f_{X_{u, 0}}(x_{u, 0}; \, \theta)
    \end{eqnarray*}\label{assumption:mle2}
\end{enumerate}
\vspace{-13mm}
Finally, the following assumptions are made about the random perturbations in lines~\ref{line:startu} and \ref{line:perturbations} of Algorithm~\ref{alg:mpif}.
Let $\mu_0$ denote the probability measure on $\big(\R^{\nshared + \Unit\nspecific}, \BorelT\big)$ that defines the distribution of the initial particle swarm, i.e., $\Theta_{1:J}^0 \overset{\iid}{\sim} \mu_0$. 
% Define $\mathcal{B}(\Theta)$ as the Borel $\sigma$-algebra on $\Theta$.
If $\{\mu_n\}_{n \geq 1}$ is a sequence of random probability measures \citep{crauel02} on $\big(\R^{\nshared + \Unit\nspecific}, \BorelT\big)$ with $\mathcal{F}_n$ denoting the corresponding filtration, then we denote $K_{\mu_n}$ to be the Markov kernel such that \linebreak $\theta \sim K_{\mu_n}(\theta', d\theta) \iff \theta \overset{dist}{=}\theta' + \addRV,$ where $\addRV|\mathcal{F}_n \sim \mu_n$.

Let $S_{\tilde{u}} = \sum_{k = 1}^{\tilde{u}} (N_{k} + 1)$. 
We define the Markov kernel as a sequence in $\nclone \in \mathbb{N}$, where $\nclone$ defines the values $(m, u, n)$ via the equation $\nclone = (\nmif-1)S_\Unit + S_{\unit-1}+n+1$, such that $K_{\nclone}(\theta_{\nclone-1}, d\theta_{\nclone}) = h_{u, n}(\theta_{\nclone}|\theta_{\nclone-1};\sigma_{u, \nmif})d\theta_{\nclone}$.
% whenever $(m-1)S_U + S_{u-1} < \nclone \leq (m-1)S_U + S_{u}$. 
Let $\{U_{\nclone}\}_{\nclone \geq 1}$ be a sequence of $\Theta$-valued random variables such that for all $\nclone \geq 1$ and sets $A_1, \ldots, A_{\nclone} \in \mathcal{B}(\Theta)$, where $\mathcal{B}(\Theta)$ is the Borel $\sigma$-algebra on $\Theta$. 
    $$
    \prob\big(\addRV_k \in A_k, k\in \{1, \ldots, \nclone\} | \mathcal{F}_{\nclone}\big) = \prod_{k = 1}^{\nclone} \mu_k(A_k). 
    $$
Then the sequence of probability measures $\{\mu_{\nclone}\}_{\nclone \geq 0}$ satisfy:
    \begin{enumerate}[label=(C\arabic*), ref=(C\arabic*)]
      \item $\mu_0\big(B_{\epsilon}(\theta)\big) > 0$ for all $\theta \in \Theta$ and $\epsilon \in (0, \infty)$.\label{assumption:kernel1}
      \item There exists a family of $(0, 1]$-valued random variables $(\Gamma^\mu_\delta)_{\delta \in (0, \infty)}$ such that, for all $\delta \in (0, \infty)$, we have $\prob\big(\inf_{\nclone \geq 1}\mu_{\nclone}\big(B_\delta(0)\big) \geq \Gamma_{\delta}^\mu\big) = 1$.\label{assumption:kernel2}
      \item $\prob\big(\inf_{\nclone \geq 1}\inf_{\theta' \in \Theta}\int_\Theta K_{\mu_{\nclone}}(\theta', d\theta) \geq \Gamma^\mu\big) = 1$ for some $(0, 1]$-valued random variable $\Gamma^\mu$.\label{assumption:kernel3}
      \item \label{assumption:kernel4} There exists a sequence of natural numbers $\{k_{\nclone}\}_{\nclone \geq 1}$ and, for all $l \in \mathbb{N}_0$, a sequence of $(0, \infty]$ valued functions $\{g_{\nclone, l}\}_{\nclone \geq 1}$ defined on $(0, \infty)$ such that \begin{enumerate} 
  \item $\lim_{\nclone \rightarrow \infty} k_{\nclone} / {\nclone} = \lim_{\nclone \rightarrow \infty} 1/k_{\nclone} = 0$, and for all $\epsilon \in (0, \infty)$, $\lim_{\nclone \rightarrow \infty}g_{\nclone, l}(\epsilon) = 0$. 
  \item For all $\nclone \geq 1$ such that $\nclone > 2k_{\nclone}$, $k^*_{\nclone} \in \{k_{\nclone}, \ldots, 2k_{\nclone}\}$, and for all $\epsilon \in (0, \infty)$,
  $$
  \frac{1}{\nclone - k^*_{\nclone}}\log \prob \bigg(\exists s \in \{k^*_{\nclone} + 1, \ldots, \nclone\}: \sum_{i = k^*_{\nclone} + 1}^s \addRV_i \notin B_{\epsilon}(0) | \mathcal{F}_{\nclone}\bigg) \leq -\frac{1}{g_{\nclone, l}(\epsilon)}.
  $$
  \item For all $\nclone > l$ and $\epsilon \in (0, \infty)$, we have 
  $$
  \frac{1}{\nclone-l}\log \prob \bigg(\sum_{i = l + 1}^{s} \addRV_i \in B_{\epsilon}(0), \forall s \in \{l + 1, \ldots, \nclone\}|\mathcal{F}_{\nclone}\bigg) \geq -g_{\nclone, l}(\epsilon).
  $$
  \end{enumerate}    
\end{enumerate}