State space models are widely used for conducting time series analysis.
Developing a state space model involves proposing mathematical equations that describe how a data-generating system evolves over time and how observations of the system are obtained.
These models are particularly useful when a scientific hypothesis about system dynamics exists, as is common when modeling ecological populations or tracking infectious disease outbreaks over time.
However, except for the simplest cases, state space models do not permit closed-form expressions of their likelihood functions, presenting challenges for inference.
This thesis presents three projects that introduce innovations in likelihood-based inference for state space models.

The first project proposes a novel approach for performing inference on Auto Regressive Moving Average (ARMA) time series models, which are formally linearly Gaussian state space models.
ARMA models are among the most frequently taught and widely used methods for time series analysis.
In this project, I demonstrate that existing algorithms and software for parameter estimation often produce sub-optimal parameter estimates with surprising frequency.
I introduce a novel random initialization algorithm designed to leverage the structure of the ARMA likelihood function to help overcome these optimization shortcomings.
Additionally, I demonstrate that profile likelihoods offer superior confidence intervals compared to those based on the Fisher information matrix---the current standard practice for ARMA modeling.

The second project presents a likelihood-based analysis of the 2010-2019 cholera outbreak in Haiti.
This work explores three distinct state space models for cholera incidence data and demonstrates the effectiveness of recently developed algorithms for performing inference in a high-dimensional setting.
A key focus of this project is to assess the strengths and limitations of using state space models to inform public health policy decisions.
Existing methodologies and workflows for this purpose are evaluated, and revised data analysis strategies that lead to better statistical fit and outcomes are presented.
For example, I demonstrate a reproducible framework for diagnosing model misspecification and subsequently developing enhancements that result in better recommendations for policy decisions.

The third project proposes a simulation-based algorithm designed to perform maximum likelihood estimation for a class of high-dimensional state space models.
This algorithm, called the Marginalized Panel Iterated Filter (MPIF), significantly enhances the capability of iterated filtering algorithms to estimate parameters for large collections of dynamically independent state space models that have shared parameters.
Improvements in parameter estimates and empirical convergence rates are achieved by addressing the issue of particle depletion that occurs when performing iterated filtering on models that have high-dimensional parameter spaces.
Theoretical support for the algorithm is provided through an analysis of iterating marginalized Bayes maps.
Additionally, asymptotic theory demonstrating the convergence of general iterated filtering algorithms for panel models without the marginalization step is presented. 
