I am deeply grateful to all of the friends, family, and mentors in my life who have helped me get to this point.
I will name some of the key contributors here, recognizing that many others have also played an important role in my journey, even if not mentioned individually.

First, I would like to express my heartfelt gratitude to my adviser, Dr. Edward L. Ionides.
I could not have asked for a better mentor, and I am immensely thankful for all you have done for me.
Beyond guiding me through my academic journey, you have profoundly influenced my beliefs and perspectives on statistics and higher education.
You have always been readily available for discussions about projects or concerns, and I have consistently looked forward to our meetings.
As I advance in my own career, I will often reflect on the exemplary standard you set as both a researcher and mentor.

I would also like to thank Dr. Aaron A. King for his mentorship. 
Software development became a much more significant part of my experience at the University of Michigan than I initially anticipated, and you played a crucial role in helping me develop good software practices. 
Whether through one-on-one interactions or by providing numerous examples of well-designed, documented, and reproducible code, your guidance has been instrumental.
You have always been available when I needed advice and were quick to offer helpful insights.

I wish to also thank the other members of my committee, Dr. Kerby Shedden and Dr. Jeffrey Regier, who have provided their advise and mentorship.
Though we did not meet regularly, I am incredibly grateful for the time you have spent helping me prepare this thesis, provide letters of recommendation, and provide general career advice.
I have looked to both of you as exemplary scholars and mentors, and you have helped shape my experience as a student more than you may realize.

I've also been fortunate to have a wonderful group of friends and classmates who have supported me throughout my PhD studies.
I would be remiss not to specifically thank members of my cohort: Prayag Chatha, Pramit Das, Luke Francisco, Kihyuk Hong, Declan McNamara, Bo Meng, Gang Qiao, Junting Wang, and Yilei Zhang.
We began the PhD program amid a global pandemic, which presented a number of unique challenges.
Together, you helped me navigate the virtual classroom environment, collaborated on homework and group projects, and provided invaluable support over the years.
This group of individuals contains many of the smartest and most talented people I have ever met, and I hope our paths will once again cross in the near future.

My family deserves a great deal of thanks for their unwavering support over the years.
To my father, you have been the best role model, showing me what it means to be a good person.
As long as I can remember, I aspired to be just like you, this perhaps being my earliest motivation to pursue a PhD.
To my mother, thank you for your endless love and support.
You have been a role model for me in countless ways. 
I am continually amazed by how you manage to balance a meaningful career, social life, personal hobbies, and a healthy lifestyle, all while prioritizing your family above everything else.
Your energy, kindness, and talents are truly unmatched.

Most importantly I would like to thank my spouse, Haylee Wheeler, who has been incredibly supportive and encouraging throughout my entire academic career.
The progress that I have made as a scholar would not have been possible without her.
Despite not having any connections to Michigan, you were willing to move across the country with me just so that I could pursue my dream of getting a PhD.
When we arrived in Ann Arbor in the midst of the pandemic, we had little opportunity to meet people or make friends; I'm am not confident that I could have endured this stage of my PhD program without you.
You have always been willing to listen to my complaints and struggles, and been with me to celebrate life's victories.
I'm so grateful for your love and support in ways that I cannot possibly express.
You are the best friend and companion I could have ever hoped for, and I am excited to share the next chapter of our lives together. 

This dissertation contains three projects described in Chapters~\ref{chpt:arima}--\ref{chpt:mpif}.
The work outlined in each of the chapters have either been published in a peer-reviewed journal (Chapter~\ref{chpt:haiti}), or are currently being prepared for publication (Chapters~\ref{chpt:arima} and \ref{chpt:mpif}). 
These projects include work that was done in conjunction with the coauthors of these papers, and I would like to explicitly thank them and highlight their contributions.
\begin{itemize}
  \item  In Chapter~\ref{chpt:arima}, my adviser Edward L. Ionides met with me weekly to discuss ideas and progress related to this project.
  He also provided valuable feedback and suggestions on how to present my algorithm and results, and was instrumental in refining the manuscript text. 
  Importantly, he was the first to notice the apparent optimization issue present in existing algorithms, which was the original motivation for this work.
  \item Chapter~\ref{chpt:haiti} was a large undertaking by several members of Dr. Ionides's research group, and I had the privilege of supervising a group of undergraduate students (AnnaElaine Rosengart, Kevin Tan, and Noah Treutle) as well as a master student (Zhuoxun (Josh) Jiang) who worked on this project. 
  All coauthors were part of weekly discussions on the project and helped refine the manuscript text.
  AnnaElaine had a major role in writing software to build Model~1, and Josh wrote code that enabled us to reproduce existing results that were obtained using Model~2.
  Kevin Tan and Noah Treutle began investigations related to the linear reductions in transmission rate for Model~1.
  This work is now published in PLoS Computational Biology \citep{wheeler24}. 
  \item Chapter~\ref{chpt:mpif} represents joint work with Dr. Ionides and Aaron Abkemeier, a classmate and coauthor of this project. 
  Aaron is an expert in Measles transmission models, and is primarily responsible for the analysis presented in Section~\ref{sec:UKmeas}.
  I am incredibly grateful for his contributions, and the chance I had to work closely with him on this project.
\end{itemize}

I would also like to acknowledge funding opportunities that have helped make this work possible.
First, I am incredibly honored to have received a Rackham Merit Fellowship (RMF) that has supported me throughout my PhD experience.
I also received funds through my Advisor from a National Science Foundation Research Training Grant (RTG) DMS-1646108, and also received a Rackham Research Grant (U086898) that helped fund the projects reported in this thesis.
