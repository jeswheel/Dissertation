% University of Michigan Dissertation LaTeX Template
% Edited: Feb 17, 2025

% Use the University of Michigan thesis class.
\documentclass[thesis]{thesis-umich}\usepackage[]{graphicx}\usepackage[]{xcolor}
% maxwidth is the original width if it is less than linewidth
% otherwise use linewidth (to make sure the graphics do not exceed the margin)
\makeatletter
\def\maxwidth{ %
  \ifdim\Gin@nat@width>\linewidth
    \linewidth
  \else
    \Gin@nat@width
  \fi
}
\makeatother

\definecolor{fgcolor}{rgb}{0.345, 0.345, 0.345}
\newcommand{\hlnum}[1]{\textcolor[rgb]{0.686,0.059,0.569}{#1}}%
\newcommand{\hlsng}[1]{\textcolor[rgb]{0.192,0.494,0.8}{#1}}%
\newcommand{\hlcom}[1]{\textcolor[rgb]{0.678,0.584,0.686}{\textit{#1}}}%
\newcommand{\hlopt}[1]{\textcolor[rgb]{0,0,0}{#1}}%
\newcommand{\hldef}[1]{\textcolor[rgb]{0.345,0.345,0.345}{#1}}%
\newcommand{\hlkwa}[1]{\textcolor[rgb]{0.161,0.373,0.58}{\textbf{#1}}}%
\newcommand{\hlkwb}[1]{\textcolor[rgb]{0.69,0.353,0.396}{#1}}%
\newcommand{\hlkwc}[1]{\textcolor[rgb]{0.333,0.667,0.333}{#1}}%
\newcommand{\hlkwd}[1]{\textcolor[rgb]{0.737,0.353,0.396}{\textbf{#1}}}%
\let\hlipl\hlkwb

\usepackage{framed}
\makeatletter
\newenvironment{kframe}{%
 \def\at@end@of@kframe{}%
 \ifinner\ifhmode%
  \def\at@end@of@kframe{\end{minipage}}%
  \begin{minipage}{\columnwidth}%
 \fi\fi%
 \def\FrameCommand##1{\hskip\@totalleftmargin \hskip-\fboxsep
 \colorbox{shadecolor}{##1}\hskip-\fboxsep
     % There is no \\@totalrightmargin, so:
     \hskip-\linewidth \hskip-\@totalleftmargin \hskip\columnwidth}%
 \MakeFramed {\advance\hsize-\width
   \@totalleftmargin\z@ \linewidth\hsize
   \@setminipage}}%
 {\par\unskip\endMakeFramed%
 \at@end@of@kframe}
\makeatother

\definecolor{shadecolor}{rgb}{.97, .97, .97}
\definecolor{messagecolor}{rgb}{0, 0, 0}
\definecolor{warningcolor}{rgb}{1, 0, 1}
\definecolor{errorcolor}{rgb}{1, 0, 0}
\newenvironment{knitrout}{}{} % an empty environment to be redefined in TeX

\usepackage{alltt}
%%% Packages included in thesis-umich class
%\RequirePackage[margin=1in,footskip=8pt,headsep=0.4cm,headheight=\baselineskip]{geometry}
%\RequirePackage{amsmath}
%\RequirePackage{amsfonts}
%\RequirePackage{amssymb}
%\RequirePackage{graphicx}
%\RequirePackage{subcaption}
%\RequirePackage{times}
%\RequirePackage{natbib}
%\RequirePackage{verbatim}
%\RequirePackage{upquote}
%\RequirePackage{textcomp}
%\RequirePackage{setspace}
%\RequirePackage{ifthen}
%\RequirePackage{soul}
%\RequirePackage{float}
%\RequirePackage{acronym}
%\RequirePackage{makeidx}
%\RequirePackage{fancyhdr}
%\RequirePackage{multicol}

% Include your packages here by including a \usepackage{<package_name>} command.
\usepackage{blindtext} % Example package which populates the ever-popular lorem ipsum text.
\usepackage{changepage}
\usepackage{marvosym} % arima
\usepackage{nameref,hyperref} % arima
\usepackage{array} % arima
\usepackage{lastpage} % arima
\usepackage{epstopdf}
\usepackage{psfrag,epsf}
\usepackage{url}
\usepackage[ruled,noline,linesnumbered]{algorithm2e}
\usepackage[table]{xcolor}
% \usepackage{mathtools}
% \DeclareMathOperator*{\argmax}{arg\,max}
% 
% \newcommand\code{\texttt}
% \newcommand\eic[1]{\textcolor{orange}{[EI: #1]}}
% \newcommand\jwc[1]{\textcolor{brown}{[JW: #1]}}
% \newcommand\TODO[1]{\textcolor{red}{[TODO: #1]}}
\newcommand\allVar{\psi}
\newcommand\allVarExt{\bar{\psi}}
\newcommand\improvedCell{\cellcolor{blue!25}}
\newcommand\R{\mathbb{R}}
\renewcommand{\emph}[1]{\textit{#1}}


% HAITI
\usepackage[nopatch=eqnum]{microtype}
\usepackage{caption}
\RequirePackage{amsthm,enumerate,xr,lmodern}
\usepackage{makecell}
\usepackage{multirow}
\usepackage{multicol}
\usepackage{bm}

%%%% parameters %%%%%%%%%%%
\newcommand\Wsat{W_{\mathrm{sat}}}
\newcommand\muIR{\mu_{IR}}
\newcommand\muEI{\mu_{EI}}
\newcommand\transmission{\beta}
\newcommand\seasAmplitude{a}
\newcommand\rainfallExponent{r}
\newcommand\muRS{\mu_{RS}}
\newcommand\vaccineEfficacy{\vartheta}
\newcommand\muBirth{\mu_S}
\newcommand\muDeath{\delta}
\newcommand\choleraDeath{\delta_{C}}
\newcommand\symptomFrac{f}
\newcommand\asymptomRelativeInfect{\epsilon}
\newcommand\asymptomRelativeShed{\epsilon_{W}}
\newcommand\Wbeta[1]{\beta_{W#1}}
\newcommand\Whur[1]{\beta_{W#1}^{hm}}
\newcommand\hHur[1]{h_{#1}^{hm}}
\newcommand\tHur{t_{hm}}
\newcommand\Iinit{I_{0,0}}
\newcommand\Einit{E_{0, 0}}
\newcommand\Wremoval{\delta_W}
\newcommand\Wshed{\mu_W}
\newcommand\mixExponent{\nu}
\newcommand\sigmaProc{\sigma_{\mathrm{proc}}}
\newcommand\reportRate{\rho}
\newcommand\obsOverdispersion{\psi}
\newcommand\phaseParm{\phi}
\newcommand\transmissionTrend{\zeta}
\newcommand\Binit{\xi}
\newcommand\vaccClass{Z}
\newcommand\vaccCounter{z}
\newcommand\modelCounter{m}
\newcommand\missing{}
\newcommand\fixed{^\dagger}
\newcommand\demography{\bullet}
\newcommand\code[1]{\texttt{#1}}
\newcommand\figTitle{\bf}
\newcommand\paramVec{\theta}
\newcommand\childReduce{q}
\newcommand\NBintercept{\alpha}
\newcommand\NBar{\beta}
\newcommand\NBsize{\varphi}

\DeclareSymbolFont{matha}{OML}{txmi}{m}{it}% txfonts
\DeclareMathSymbol{\varv}{\mathord}{matha}{118}
\DeclareMathOperator*{\argmax}{arg\,max}
\newcommand\myeqref[1]{(\ref{#1})}
\newcommand{\blind}{1}

%% customized math macros
\newcommand\seq[2]{{#1}\!:\!{#2}}
% \newcommand\R{\mathbb{R}}
\newcommand\Var{\mathrm{Var}}
\newcommand\var{\Var}
\newcommand\Cov{\mathrm{Cov}}
\newcommand\cov{\Cov}
\newcommand\iid{\mathrm{iid}}
\newcommand\dist{d}
\def\lik{L}
\def\loglik{\ell}

\newcolumntype{t}{>{\tiny}c}




%--- Set the font styles -----
% As of this writing, Rackham does not have strict font requirements, but they
% suggest "standard fonts" such as Times, Arial, or Times New Roman.
% In practice, they allow the default LaTeX font (Computer Modern).
% The fonts available to you depend on the typesetting engine you use, e.g.
% pdflatex (the overleaf default), xelatex, or lualatex.
% Set the font to whatever you like, as long as it's compatible with the tex
% engine you're using.
% More information from Overleaf on font selection with:
%     - pdflatex: https://www.overleaf.com/learn/latex/Font_typefaces
%     - xelatex: https://www.overleaf.com/learn/latex/XeLaTeX
% See examples below depending on your engine and uncomment the usepackage
% commands and any subsequent commands as needed.
%-------------------------------------------------------------------------------
%--- pdflatex -----
%\usepackage{times}
%\usepackage{newtxtext}
%\usepackage{newtxmath}
%-------------------------------------------------------------------------------
%--- xelatex / lualatex -----
%\usepackage{fontspec}
%\setromanfont{Times New Roman}
%\setsansfont{Arial}
%\setmonofont{Courier New}
%-------------------------------------------------------------------------------

% If you are using the alpha bibliography style, keep these next three lines in your preamble, so that the references are left-aligned; or, you can comment it out and see what happens
\makeatletter
\renewcommand{\@biblabel}[1]{[#1]\hfill}
\makeatother

% Title of the thesis
\title{Innovations in Likelihood-Based Inference for State Space Models}

% Author name
\author{Jesse Wheeler}

% Specify the author's legal name, if different from the preferred name.
%\legalname{Your Legal Name}

% Department
\department{Statistics}

% Year of completion
\year=2025

% Author email
\email{jeswheel@umich.edu}

% Author ORCID iD
\orcid{0000-0003-3941-3884}

% Frontispiece
\frontispiece{\includegraphics[width=4in]{front_materials/frontispiece.png}}

% Default style for front pages
\frontpagestyle{7} % 7 is preferred by Rackham, but should be set individually for each front page

% Dedication (the input [7] determines the style -- 7 is Rackham's preferred style)
% \hidededication{%
% \dedication[7]{Dedicated to my family: past, present, and future.}

% Acknowledgments (the input [7] determines the style -- 7 is Rackham's preferred style)
\acknowledgments[7]{I am deeply grateful to all of the mentors in my life who have helped me get to this point.
I am particularly grateful for my advisor, Dr. Edward L. Ionides, who has provided invaluable support for me throughout my academic journey and has helped shape my beliefs about statistics and higher education.
I wish to also thank the other members of my committee who have provided their advise and mentorship. 
Dr. Aaron A. King 

Most importantly I would like to thank my spouse, Haylee Wheeler, who has been incredibly supportive and encouranging though my entire academic career.
The progress that I have made as a scholar would not have been possible without her.
You have always been willing to listen to my complaints and struggles, and been with me to celebrate life's victories.
}

% Preface
% \preface[7]{\input{front_materials/preface}}

% Committee
\committee{ %
Professor Edward L. Ionides, Chair \\
Professor Aaron A. King \\
Assistant Professor Jeffrey Regier \\
Professor Kerby Shedden \\
}

% Chair must be entered separately for formatting reasons.
\chair{Professor Edward L. Ionides}
%\cochair{Co-chair One \& Co-chair Two}


% Definition of any acronyms used.
% To add an acronym, add an \acro{}{} command on a new line within the \acronyms{} command. For \acro, field 1 is the acronym and field 2 is the corresponding expression. For example: \acro{TLA}{Three Letter Acronym}.
% \acronyms{
%     \acro{TLA}{Three Letter Acronym}
%     \acro{SOA}{Some Other Acronym}
% }

% Definition of any symbols used.
% To add an symbol, add an \item command with the symbol inside a [] bracket followed by explanations
% \symbols{
%     \item [$\alpha$] The greek letter alpha.
%     \item [$\Gamma$] \blindtext
% }

% Commands to hide or show lists of figures, tables, etc.
% To hide a list, change the word "show" in the command to "hide".
\showlistoftables
\showlistofprograms
\showlistofappendices
% \showlistofacronyms
% \hidelistofsymbols


% Some abstract text
\abstract{
State space models are widely used for conducting time series analysis.
Developing a state space model involves proposing mathematical equations that describe how a data-generating system evolves over time and how observations of the system are obtained.
These models are particularly useful when a scientific hypothesis about system dynamics exists, as is common when modeling ecological populations or tracking infectious disease outbreaks over time.
However, except for the simplest cases, state space models do not permit closed-form expressions of their likelihood functions, presenting challenges for inference.
This thesis presents three projects that introduce innovations in likelihood-based inference for state space models.

The first project proposes a novel approach for performing inference on Auto Regressive Moving Average (ARMA) time series models, which are formally linearly Gaussian state space models.
ARMA models are among the most frequently taught and widely used methods for time series analysis.
In this project, I demonstrate that existing algorithms and software for parameter estimation often produce sub-optimal parameter estimates with surprising frequency.
I introduce a novel random initialization algorithm designed to leverage the structure of the ARMA likelihood function to help overcome these optimization shortcomings.
Additionally, I demonstrate that profile likelihoods offer superior confidence intervals compared to those based on the Fisher information matrix---the current standard practice for ARMA modeling.

The second project presents a likelihood-based analysis of the 2010-2019 cholera outbreak in Haiti.
This work explores three distinct state space models for cholera incidence data and demonstrates the effectiveness of recently developed algorithms for performing inference in a high-dimensional setting.
A key focus of this project is to assess the strengths and limitations of using state space models to inform public health policy decisions.
Existing methodologies and workflows for this purpose are evaluated, and revised data analysis strategies that lead to better statistical fit and outcomes are presented.
For example, I demonstrate a reproducible framework for diagnosing model misspecification and subsequently developing enhancements that result in better recommendations for policy decisions.

The third project proposes a simulation-based algorithm designed to perform maximum likelihood estimation for a class of high-dimensional state space models.
This algorithm, called the Marginalized Panel Iterated Filter (MPIF), significantly enhances the capability of iterated filtering algorithms to estimate parameters for large collections of dynamically independent state space models that have shared parameters.
Improvements in parameter estimates and empirical convergence rates are achieved by addressing the issue of particle depletion that occurs when performing iterated filtering on models that have high-dimensional parameter spaces.
Theoretical support for the algorithm is provided through an analysis of iterating marginalized Bayes maps.
Additionally, asymptotic theory demonstrating the convergence of general iterated filtering algorithms for panel models without the marginalization step is presented. 

}
%\hideabstractpagenumber

%% DOCUMENT AREA
\IfFileExists{upquote.sty}{\usepackage{upquote}}{}
\begin{document}

% University of Michigan Dissertation LaTeX Template
% Edited: Feb 17, 2025

% Use the University of Michigan thesis class.
\documentclass[thesis]{thesis-umich}
%%% Packages included in thesis-umich class
%\RequirePackage[margin=1in,footskip=8pt,headsep=0.4cm,headheight=\baselineskip]{geometry}
%\RequirePackage{amsmath}
%\RequirePackage{amsfonts}
%\RequirePackage{amssymb}
%\RequirePackage{graphicx}
%\RequirePackage{subcaption}
%\RequirePackage{times}
%\RequirePackage{natbib}
%\RequirePackage{verbatim}
%\RequirePackage{upquote}
%\RequirePackage{textcomp}
%\RequirePackage{setspace}
%\RequirePackage{ifthen}
%\RequirePackage{soul}
%\RequirePackage{float}
%\RequirePackage{acronym}
%\RequirePackage{makeidx}
%\RequirePackage{fancyhdr}
%\RequirePackage{multicol}

% Include your packages here by including a \usepackage{<package_name>} command.
\usepackage{blindtext} % Example package which populates the ever-popular lorem ipsum text.
\usepackage{changepage}
\usepackage{marvosym} % arima
\usepackage{nameref,hyperref} % arima
\usepackage{array} % arima
\usepackage{lastpage} % arima
\usepackage{epstopdf}
\usepackage{psfrag,epsf}
\usepackage{url}
\usepackage[ruled,noline,linesnumbered]{algorithm2e}
\usepackage[table]{xcolor}
% \usepackage{mathtools}
% \DeclareMathOperator*{\argmax}{arg\,max}
% 
% \newcommand\code{\texttt}
% \newcommand\eic[1]{\textcolor{orange}{[EI: #1]}}
% \newcommand\jwc[1]{\textcolor{brown}{[JW: #1]}}
% \newcommand\TODO[1]{\textcolor{red}{[TODO: #1]}}
\newcommand\allVar{\psi}
\newcommand\allVarExt{\bar{\psi}}
\newcommand\improvedCell{\cellcolor{blue!25}}
\newcommand\R{\mathbb{R}}
\renewcommand{\emph}[1]{\textit{#1}}


% HAITI
\usepackage[nopatch=eqnum]{microtype}
\usepackage{caption}
\RequirePackage{amsthm,enumerate,xr,lmodern}
\usepackage{makecell}
\usepackage{multirow}
\usepackage{multicol}
\usepackage{bm}

%%%% parameters %%%%%%%%%%%
\newcommand\Wsat{W_{\mathrm{sat}}}
\newcommand\muIR{\mu_{IR}}
\newcommand\muEI{\mu_{EI}}
\newcommand\transmission{\beta}
\newcommand\seasAmplitude{a}
\newcommand\rainfallExponent{r}
\newcommand\muRS{\mu_{RS}}
\newcommand\vaccineEfficacy{\vartheta}
\newcommand\muBirth{\mu_S}
\newcommand\muDeath{\delta}
\newcommand\choleraDeath{\delta_{C}}
\newcommand\symptomFrac{f}
\newcommand\asymptomRelativeInfect{\epsilon}
\newcommand\asymptomRelativeShed{\epsilon_{W}}
\newcommand\Wbeta[1]{\beta_{W#1}}
\newcommand\Whur[1]{\beta_{W#1}^{hm}}
\newcommand\hHur[1]{h_{#1}^{hm}}
\newcommand\tHur{t_{hm}}
\newcommand\Iinit{I_{0,0}}
\newcommand\Einit{E_{0, 0}}
\newcommand\Wremoval{\delta_W}
\newcommand\Wshed{\mu_W}
\newcommand\mixExponent{\nu}
\newcommand\sigmaProc{\sigma_{\mathrm{proc}}}
\newcommand\reportRate{\rho}
\newcommand\obsOverdispersion{\psi}
\newcommand\phaseParm{\phi}
\newcommand\transmissionTrend{\zeta}
\newcommand\Binit{\xi}
\newcommand\vaccClass{Z}
\newcommand\vaccCounter{z}
\newcommand\modelCounter{m}
\newcommand\missing{}
\newcommand\fixed{^\dagger}
\newcommand\demography{\bullet}
\newcommand\code[1]{\texttt{#1}}
\newcommand\figTitle{\bf}
\newcommand\paramVec{\theta}
\newcommand\childReduce{q}
\newcommand\NBintercept{\alpha}
\newcommand\NBar{\beta}
\newcommand\NBsize{\varphi}

\DeclareSymbolFont{matha}{OML}{txmi}{m}{it}% txfonts
\DeclareMathSymbol{\varv}{\mathord}{matha}{118}
\DeclareMathOperator*{\argmax}{arg\,max}
\newcommand\myeqref[1]{(\ref{#1})}
\newcommand{\blind}{1}

%% customized math macros
\newcommand\seq[2]{{#1}\!:\!{#2}}
% \newcommand\R{\mathbb{R}}
\newcommand\Var{\mathrm{Var}}
\newcommand\var{\Var}
\newcommand\Cov{\mathrm{Cov}}
\newcommand\cov{\Cov}
\newcommand\iid{\mathrm{iid}}
\newcommand\dist{d}
\def\lik{L}
\def\loglik{\ell}

\newcolumntype{t}{>{\tiny}c}




%--- Set the font styles -----
% As of this writing, Rackham does not have strict font requirements, but they
% suggest "standard fonts" such as Times, Arial, or Times New Roman.
% In practice, they allow the default LaTeX font (Computer Modern).
% The fonts available to you depend on the typesetting engine you use, e.g.
% pdflatex (the overleaf default), xelatex, or lualatex.
% Set the font to whatever you like, as long as it's compatible with the tex
% engine you're using.
% More information from Overleaf on font selection with:
%     - pdflatex: https://www.overleaf.com/learn/latex/Font_typefaces
%     - xelatex: https://www.overleaf.com/learn/latex/XeLaTeX
% See examples below depending on your engine and uncomment the usepackage
% commands and any subsequent commands as needed.
%-------------------------------------------------------------------------------
%--- pdflatex -----
%\usepackage{times}
%\usepackage{newtxtext}
%\usepackage{newtxmath}
%-------------------------------------------------------------------------------
%--- xelatex / lualatex -----
%\usepackage{fontspec}
%\setromanfont{Times New Roman}
%\setsansfont{Arial}
%\setmonofont{Courier New}
%-------------------------------------------------------------------------------

% If you are using the alpha bibliography style, keep these next three lines in your preamble, so that the references are left-aligned; or, you can comment it out and see what happens
\makeatletter
\renewcommand{\@biblabel}[1]{[#1]\hfill}
\makeatother

% Title of the thesis
\title{Innovations in Likelihood-Based Inference for State Space Models}

% Author name
\author{Jesse Wheeler}

% Specify the author's legal name, if different from the preferred name.
%\legalname{Your Legal Name}

% Department
\department{Statistics}

% Year of completion
\year=2025

% Author email
\email{jeswheel@umich.edu}

% Author ORCID iD
\orcid{0000-0003-3941-3884}

% Frontispiece
\frontispiece{\includegraphics[width=4in]{front_materials/frontispiece.png}}

% Default style for front pages
\frontpagestyle{7} % 7 is preferred by Rackham, but should be set individually for each front page

% Dedication (the input [7] determines the style -- 7 is Rackham's preferred style)
% \hidededication{%
% \dedication[7]{Dedicated to my family: past, present, and future.}

% Acknowledgments (the input [7] determines the style -- 7 is Rackham's preferred style)
\acknowledgments[7]{I am deeply grateful to all of the mentors in my life who have helped me get to this point.
I am particularly grateful for my advisor, Dr. Edward L. Ionides, who has provided invaluable support for me throughout my academic journey and has helped shape my beliefs about statistics and higher education.
I wish to also thank the other members of my committee who have provided their advise and mentorship. 
Dr. Aaron A. King 

Most importantly I would like to thank my spouse, Haylee Wheeler, who has been incredibly supportive and encouranging though my entire academic career.
The progress that I have made as a scholar would not have been possible without her.
You have always been willing to listen to my complaints and struggles, and been with me to celebrate life's victories.
}

% Preface
% \preface[7]{\input{front_materials/preface}}

% Committee
\committee{ %
Professor Edward L. Ionides, Chair \\
Professor Aaron A. King \\
Assistant Professor Jeffrey Regier \\
Professor Kerby Shedden \\
}

% Chair must be entered separately for formatting reasons.
\chair{Professor Edward L. Ionides}
%\cochair{Co-chair One \& Co-chair Two}


% Definition of any acronyms used.
% To add an acronym, add an \acro{}{} command on a new line within the \acronyms{} command. For \acro, field 1 is the acronym and field 2 is the corresponding expression. For example: \acro{TLA}{Three Letter Acronym}.
% \acronyms{
%     \acro{TLA}{Three Letter Acronym}
%     \acro{SOA}{Some Other Acronym}
% }

% Definition of any symbols used.
% To add an symbol, add an \item command with the symbol inside a [] bracket followed by explanations
% \symbols{
%     \item [$\alpha$] The greek letter alpha.
%     \item [$\Gamma$] \blindtext
% }

% Commands to hide or show lists of figures, tables, etc.
% To hide a list, change the word "show" in the command to "hide".
\showlistoftables
\showlistofprograms
\showlistofappendices
% \showlistofacronyms
% \hidelistofsymbols


% Some abstract text
\abstract{
State space models are widely used for conducting time series analysis.
Developing a state space model involves proposing mathematical equations that describe how a data-generating system evolves over time and how observations of the system are obtained.
These models are particularly useful when a scientific hypothesis about system dynamics exists, as is common when modeling ecological populations or tracking infectious disease outbreaks over time.
However, except for the simplest cases, state space models do not permit closed-form expressions of their likelihood functions, presenting challenges for inference.
This thesis presents three projects that introduce innovations in likelihood-based inference for state space models.

The first project proposes a novel approach for performing inference on Auto Regressive Moving Average (ARMA) time series models, which are formally linearly Gaussian state space models.
ARMA models are among the most frequently taught and widely used methods for time series analysis.
In this project, I demonstrate that existing algorithms and software for parameter estimation often produce sub-optimal parameter estimates with surprising frequency.
I introduce a novel random initialization algorithm designed to leverage the structure of the ARMA likelihood function to help overcome these optimization shortcomings.
Additionally, I demonstrate that profile likelihoods offer superior confidence intervals compared to those based on the Fisher information matrix---the current standard practice for ARMA modeling.

The second project presents a likelihood-based analysis of the 2010-2019 cholera outbreak in Haiti.
This work explores three distinct state space models for cholera incidence data and demonstrates the effectiveness of recently developed algorithms for performing inference in a high-dimensional setting.
A key focus of this project is to assess the strengths and limitations of using state space models to inform public health policy decisions.
Existing methodologies and workflows for this purpose are evaluated, and revised data analysis strategies that lead to better statistical fit and outcomes are presented.
For example, I demonstrate a reproducible framework for diagnosing model misspecification and subsequently developing enhancements that result in better recommendations for policy decisions.

The third project proposes a simulation-based algorithm designed to perform maximum likelihood estimation for a class of high-dimensional state space models.
This algorithm, called the Marginalized Panel Iterated Filter (MPIF), significantly enhances the capability of iterated filtering algorithms to estimate parameters for large collections of dynamically independent state space models that have shared parameters.
Improvements in parameter estimates and empirical convergence rates are achieved by addressing the issue of particle depletion that occurs when performing iterated filtering on models that have high-dimensional parameter spaces.
Theoretical support for the algorithm is provided through an analysis of iterating marginalized Bayes maps.
Additionally, asymptotic theory demonstrating the convergence of general iterated filtering algorithms for panel models without the marginalization step is presented. 

}
%\hideabstractpagenumber

%% DOCUMENT AREA
\begin{document}

\input{chapters/chapter01_introduction}
% Place your additional chapters here using the \input{} command
\input{chapters/example_chapter_99}

% Appendices
\appendix
\input{appendices/example_appendix_01}
\input{appendices/example_appendix_02}

\bibliographystyle{plain}

% Give this command the relative path to the .bib file.
\bibliography{references.bib}

\end{document}

\input{chapters/arima2/child}
\input{chapters/haiti/child}
% Place your additional chapters here using the \input{} command
\input{chapters/example_chapter_99}

% Appendices
\appendix
\input{appendices/example_appendix_01}
\input{appendices/example_appendix_02}

\bibliographystyle{plain}

% Give this command the relative path to the .bib file.
% \bibliography{references.bib}

% References
\bibliography{references}

\end{document}
